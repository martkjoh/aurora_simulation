\documentclass{article}
\usepackage{physics}
\usepackage{graphicx}
\usepackage{url}

\graphicspath{{./figs/}}

\title{Semester assigment ElMag - Simulation of aurora}
\author{Martin Johsnrud}

\begin{document}
    \maketitle

    Auroras come from charged particles launch into outer space by sun storms beeing guided by the magnetic field of the earth down in the atmosphere. This project simulates how the path of these charged particles are affected by magnetic field using a Runge-Kutta method.
    
    \section*{Parametres}
        This problem has many parametres, $q, M, R_{\odot}, m, \mu_0$, which is respectivley the particle charge and mass, the earths mass and magnetic dipole moment and the magnetic peremability of space. They span many orders of magnitude, which is inconvinient in numerical solution. By reducing them to 4 characteristic sizes,  length, magnetic flux density, time and energy, they can be used as convenien units fitting the problem. They are
        \begin{equation*}
            x_0 = R_{\odot}, \quad  
            B_0 = \frac{\mu_0 m}{x_0^3}, \quad 
            t_0 = \frac{M}{q}\frac{5\cdot 10^5}{B_0},  \quad 
            E_0 = M\frac{x_0^2}{t_0^2}.
        \end{equation*}
        A numeric factor has been added to the time, so that $x_0 / t_0$ is about speed of a typical coronal ejection. \footnote[1]{\url{https://en.wikipedia.org/wiki/Coronal_mass_ejection#physical_properties}} The values for these are, as some of them do not have permanet and definet values, roughly 
        \begin{align*}
            & x_0 = 6.47 \cdot 10^6 \textrm{m}, \quad
            B_0 = \frac{1.26 \cdot 10^{-6} \, 8 \cdot 10^{22}}{x_0^3} = 3.7 \cdot 10^{-4} \frac{\textrm{kg}}{\textrm{s}^{2} \, \textrm{A}} \\
            &t_0 = \frac{1.7 \cdot 10^{-27}}{1.6\cdot10^{-19}}\frac{5\cdot 10^5}{B_0} = 14 \textrm{s}, \quad 
            E_0 = M\frac{x_0^2}{t_0^2} = 3.6 \cdot 10^{-16} \frac{\textrm{kg} \, \textrm{m}^2}{\textrm{s}^2}
        \end{align*}
        These will be used as units through the exercise.

    \section*{Magnetic field}

    \begin{figure}
        \centering    
        \includegraphics[width=\textwidth]{b_field_2D}
        \caption{The earths magnetic field, in the $xz$- and $yz$-plane.  }
        \label{B-field}
    \end{figure}

    \paragraph{}
    We can model the earths magnetic field as a diploe. Let the x-axis point from the centre of the earth towards the sun, the y-axis roughly paralel to the earths orbit, and then the z-axis northwards. With $\hat m$ as the unit vector of earths magnetic dipole-moment, the magnetic field of the earth is then given by
        \begin{equation}
            B_j = \frac{1}{4\pi} \frac{3\hat m_i \hat x_i \hat x_j - \hat m_j}{(r/x_0)^3},
        \end{equation}
        where
        \begin{equation*}
            r = \sqrt{x_ix_i}, \quad \hat x_i = x_i / r.
        \end{equation*}

        Figure \ref{B-field} shows the magnetic field as seen from the eraths orbit, and on the other side than the sun. 

    \section*{Charged particles}

    \begin{figure}
        \centering
        \vspace{-50px}
        \includegraphics[width=\textwidth ]{charged_particles_2D}
        \caption{Charged particles aproaching earth from different starting points. Evne though the lines may overlap with the earth, it does not mean that they have hit the ground, as the graph only contains two dimensions.}
        \label{Charged particles}
    \end{figure}

    \begin{figure}
        \centering
        \vspace{-50px}
        \includegraphics[width=\textwidth ]{charged_particles_zoom}
        \caption{A closer look at the top paths}
        \label{Charged particles zoom}
    \end{figure}

        Let $ \hat e_i B_i(x_j)$ be the earths magnetic field. The magnetic force on a particle with charge $q$, position $\hat e_i x_i(t)$ and veloxity $\hat e_i \dot x_i (t)$ is then
        \begin{equation}
            F_i = \epsilon_{ijk} \dot x_j B_k.
        \end{equation}
        By newtons second law, we get the equation of motion for the particle,
        \begin{equation}
            \ddot x_i = \epsilon_{ijk} \dot x_j B_k
        \end{equation}
        Rewriting to a first order set of equations, we get
        \begin{align*}
            \dv{t} y = \dv{t} 
            \begin{pmatrix}
                x_i \\
                \dot x_i
            \end{pmatrix}
            = 
            \begin{pmatrix}
                \dot x_i \\
                \epsilon_{ijk} \dot x_j B_k
            \end{pmatrix}
            = f(y).
        \end{align*}
        This can then be solved numerically by a runge kutta method. This i show in figure \ref{Charged particles}, where several particles with different starting positions are simulated as they aproach the earth, and is deflected by the magnetic field of the earth. The only particels that are able to reach down to earths surface are particles apraoching from higer up, explaining why we only see auroras near the poles.

    \section*{Accuracy}
    \begin{figure}
        \centering
        \includegraphics[width = \textwidth]{relative_error_energy}
        \label{Relative error}
        \caption{The relative shifts in energy, in comparison with the starting values. Particle 8 has the largest value, $2.036 \cdot 10^{-5}$}
    \end{figure}

    To thest the accuracy of the numerical scheme, one can use the fact tha magnetic forces never does any work on particles, as they always are perpendicular to the velocity of a particle. The potential energy of the particle should thus be conserved, as the simultation does not take into account friction, nor electromagnetic radiation. Kinetic energy is given by
    $$
        E = \frac{1}{2}\dot x^2,
    $$
    in units of $E_0$. Relative error,
    $$
        \Delta E(t) = \frac{|E(t) - E(0)|}{E(0)},
    $$
    is thus a usefull cuantity for evaluating the precission of the simulations. Figure \ref{Relative error} shows this for all simulated paths.

\end{document}